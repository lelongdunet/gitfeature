This document describe the development workflow which defines the way features implementation and bugfix are made using \textbf{git-feature}. This workflow can apply to small or medium team projects whatever the programming language. Git is supposed to be used as the version control system. This document describes various steps in the development of each feature and how to manage branches in which commits are made and how they are included in the main development branch.

We focus here on the development and feature branches. Each new feature or bugfix is supposed to have its own branch. To work efficiently with the descibed workflow you need the \texttt{git-feature} script and hook scripts that are associated with it. Rebase is mainly used in this workflow allowing to have a linear history. Since it can be a hazardous operation when misused, precise rules must be followed. When properly applied, these rules allow to efficiently track each feature from its creation to its integration in the main development branch.

\section{Definitions and global presentation of the workflow}

\begin{figure}[h]
 \centering
 \includegraphics[width=.75\textwidth]{./img/workflow.pdf}
 % workflow.pdf: 357x336 pixel, 72dpi, 12.59x11.85 cm, bb=0 0 357 336
 \caption{Main steps of features development cycle}
 \label{fig:workflow}
\end{figure}

The described workwflow lies on some key concepts that needs to be defined :
\begin{itemize}
 \item \textbf{Development reference branch} or the main development branch, is the branc containing all validated commits and never must be rebased. It is subsequently referred as \texttt{\DEVREF}. New feature branches are supposed to be based on this branch which is a stable point.
 \item \textbf{Draft feature branches} are branches in which are commited changes related to each feature. They are created in git as \texttt{draft/featurename} 
 \item \textbf{Final feature branches} are branches created at the finalization step, once features are supposed to be working and ready for integration in \DEVREF. They are created in git as \texttt{final/featurename}
 \item \textbf{Dependant feature} is a feature that needs commits of another feature wich is not yet in \DEVREF.
 \item \textbf{Start branch} are branches used in dependant features as a marker in wich you are not supposed to commit. They mark the first commit of a feature when it is not just after \DEVREF. They are created in git as \texttt{start/featurename}
 \item \textbf{Updated feature} : A feature is updated when it is succesfully rebased on \DEVREF~so that it contains all its commits. In the case of dependant features, it must be rebased on the draft or final branch of the previous feature.
\end{itemize}

The most used branch during feature implementation is \texttt{draft/featurename}, you will commit all your changes in it. You are free to use various temporary branches to commit in you local repository in order to test different solutions or anything else. But once, your work is done, the draft branch should be pushed. The final branch is created by the end and you are only supposed to alter it using rebase. You rarely have to take care of the start branch since it is automatically managed by the git feature script.


Main steps of the workflow are shown figure \ref{fig:workflow}.


\section{Using \texttt{git-feature} script}

Once the script is installed on your system, you get a new git command :
\begin{lstlisting}
git feature
\end{lstlisting}
This single command should show the help. To properly work, git-feature must first be initialized using
\begin{lstlisting}
git feature -a init
\end{lstlisting}
This inialization is at the repository level so it must be run for each individual repository. It install commit and rebase hooks and set some configuration parameters. You will be asked for your main developement branch and for the public repository in which you push your changes.

You should also see some alias appear if you use shell completion
\begin{lstlisting}
git feat<tab><tab>
feat            featfinalize    featmove        feattest        featview
featclear       featintegrate   featpush        featupdate
featco          featlist        featreview      feature
\end{lstlisting}
Here is a brief description of most important of these commands :
\begin{itemize}
    \item \texttt{feature} : is the script itself and can be used to create new feature
    \item \texttt{feat} : is a shortcut for \texttt{git feature -a} and can be followed by a set of advanced commands
    \item \texttt{featupdate} : can be used after a fetch to update a feature ( a rebase is performed on \DEVREF~or the previous feature if it dependant )
    \item \texttt{featpush} : is used to push a feature on the remote. Recall that all branches (draft, final and start) must be pushed so this command avoid doing this by hand. If \texttt{all} is specified instead of a feature all feaures already pushed and modified since last push are updated.
    \item \texttt{featfinalize} : create the final branch of a feature.
    \item \texttt{featintegrate} : integrate the specified feature in the current branch of the repository. The current branch must be the \DEVREF or a temporary branch that will be merged into it. An error is raised if current branch is a feature branch.
    \item \texttt{featlist} : List all active features with their state. By default only local features are displayed but one remote name can be specified. If \texttt{all} is specified, all active features are listed.
    \item \texttt{featco} : Do the same as \texttt{git checkout} except a feature name instead of the full branch name can be directly specified. If a feature is
    specified, the current working branch of this feature is checkout.
    \item \texttt{featview} : show in gitk branches related to features related to the specified remote. By default features you are working on are selected.
 If \texttt{all} is specified instead of a remote name, all active features are shown.
    \item \texttt{featreview} : launch an interactive rebase on the right set of commits for a given feature
    \item \texttt{featmove} : this command should only be used sometimes for dependant features. It safely moves a feature branch using rebase.
    \item \texttt{featclose} : is used to delete all feature branch. This can be run only when the feature is included in \DEVREF.
    \item \texttt{feattest} : allows to easily merge a set of selected of feature branches into a test branch.
\end{itemize}
Most of these commands apply on a specific feature. Eather this feature can be specified as the last commandline argument or the current git HEAD is used.

\section{Development cycle for an independant feature}

To create and work in a new feature run
\begin{lstlisting}
git feature newfeature
\end{lstlisting}
You should be now in the draft branch of your feature. Then you can implement and commit your change a usual with git. Once feature is completed or if you just need to share your work (in progress) on the public repository you have first to check your feature is updated.
\begin{lstlisting}
git fetch --all
git featupdate
\end{lstlisting}
Check you have no conflicts, if you have, it is now time to solve them. You can perform a last check of your commits before push
\begin{lstlisting}
git featreview
\end{lstlisting}
and then push
\begin{lstlisting}
git featpush
\end{lstlisting}

Once your changes are accepted for integration or just some minor modifications are asked, you can finalize your branch test it a last time and push. Do not forget to check your feature is updated with last state of \DEVREF~ before.
\begin{lstlisting}
git fetch --all
git featupdate
git featfinalize
#Test your feature a last time
git featpush
\end{lstlisting}

Some time, severall people may need to share a feature to work on. If you can avoid it, this is better. But if you have to, it is possible, but in that case you must never EDIT or SQUASH commits that are pushed until finalization step. If you need to squash commits together you can just move them and begin the comment with '\texttt{TOSQUASH : }'.

\section{Development cycle for dependant features}

\begin{lstlisting}
\end{lstlisting}

\begin{lstlisting}
\end{lstlisting}

\begin{lstlisting}
\end{lstlisting}

Not completed yet.
