%-------------PDF-------------------------------------------------------------

%Passage au PDF de qualité
\usepackage[pdftex]{graphicx, color}       % insertion images et couleurs
%\DeclareGraphicsExtensions{.jpg,.png,.pdf}          % Formats d'images
%\pdfcompresslevel=9
\usepackage{pslatex}                               % Polices PDF, moins lourdes et non bitmap


\usepackage[pdftex]{thumbpdf}                   % vignettes sur acrobat 5.0 ou sup
\usepackage[francais]{layout}

\usepackage[pdftex,                             %   Paramétrage de la navigation
    bookmarks         = true,                       % Signets
    bookmarksnumbered = true,                       % Signets numérotés
    pdfpagemode       = None,                   % Signets/vignettes fermé à l'ouverture
    pdfstartview      = FitH,                   % La page prend toute la largeur
    pdfpagelayout     = OneColumn,              % Vue par page
    colorlinks        = false,                   % Liens en couleur
    urlcolor          = blue                       % Couleur des liens externes
    ]{hyperref}%                                % Utilisation de HyperTeX

\usepackage{pdfpages}                               % permet d'inclure des fichiers entiers pdf


%-------------PACKAGES---------------------------------------------------------
\usepackage{fancyhdr}                               % entete et pied de pages
\usepackage[Conny]{fncychap}												%Présentation des chapitres
																		%Options: Sonny, Lenny, Glenn, Conny, Rejne and Bjarne
\usepackage[outerbars]{changebar}                   % positionnement barre en marge externe
\usepackage{makeidx}                               % Indexation du document
\usepackage{multicol}                               % gestion plusieurs colonnes
\usepackage{a4wide}                                 % utilisation de toute la page A4
\usepackage{lastpage}                               % avoir la derniere page
\usepackage{verbatim}                               % Texte non interprété
\usepackage{pifont}					    %Caracteres additionnels (pifont)
\usepackage{stmaryrd}
\usepackage{textcomp}
\usepackage{array}
\usepackage{tabularx}
\usepackage{booktabs}
\usepackage{colortbl}
\usepackage{tikz}


%-------------DECLARATIONS6----------------------------------------------------
\definecolor{gris25}{gray}{0.75}

\newcommand{\boite}[1]
{\begin{minipage}{5cm}\smallskip#1\smallskip\end{minipage}}
\newcommand{\link}[1]
{\href{#1}{#1}}

%%%% -- Abréviations latines -- %%%%
\newcommand{\cf}{\textit{c.f.} : } %confer
\newcommand{\ie}{\textit{i.e.} : } %id est
\newcommand{\eg}{\textit{e.g.} : } %exempli gratia
\newcommand{\ehgo}{\textit{e.h.g.o} : } %et hoc genus omne
\newcommand{\etc}{\textit{etc.} : } %et cetera
\newcommand{\id}{\textit{id.}, } %idem
\newcommand{\idq}{\textit{i.q.} } %idem quo
\newcommand{\IF}{\textit{in fine} } %in fine : a la fin
%\newcommand{\if}{\textit{i.f.} } %in fine
\newcommand{\pa}{\textit{p.a.} : } %par annum : par an
\newcommand{\ps}{\textit{p.s.} : } %post scriptum
\newcommand{\qv}{\textit{q.v.} : } %quod vide : vu autrepart
\newcommand{\qqv}{\textit{qq.v.} : } %quae vide : pluriel
\newcommand{\vs}{\textit{v.s.} } %versus
\newcommand{\viceversa}{\textit{vice versa} } %et réciproquement

%Insertion d'espaces dans les listes
\newcommand{\itemsp}{\medskip \item }

\newenvironment{Titre}[1][15cm]{%
\begin{center}
\newlength{\LigLen}
\setlength{\LigLen}{#1}
\rule[1ex]{\LigLen}{0.5mm}\\
\mbox{}}{%
\par\rule[1ex]{\LigLen}{0.5mm}
\end{center}}

%%Definition des envirronements type 'theoremes'
\newtheorem{rem}{Remarque}[section]
%%Mise en place d'index
\usepackage{makeidx}
\makeindex

%%%% Macro pour mise en valeur de première lettre d'un paragraphe %%%%
%%Commande \cappar à placer avant la première lettre
%Liste des familles : yswab, ygoth, yfrak, cmbx12
\font\capfont= cmbx12 at 50 pt % = Famille de police 'at' Taille

\newbox\capbox \newcount\capl \def\a{A}
\def\docappar{\medbreak\noindent\setbox\capbox\hbox{%
\capfont\a\hskip0.15em}\hangindent=\wd\capbox%
\capl=\ht\capbox\divide\capl by\baselineskip\advance\capl by1%
\hangafter=-\capl%
\hbox{\vbox to8pt{\hbox to0pt{\hss\box\capbox}\vss}}}
\def\cappar{\afterassignment\docappar\noexpand\let\a }
%%%% fin macro %%%%

%-------------CONFIGURATION LISTINGS ----------------------------------------------------

\usepackage{listings}																%Insertion de codes sources

\newcommand{\lstconfigcpp}
{
    \lstset{language=C++}
    \lstset{% general command to set parameter(s)
        %-----Police et coloration syntaxique
            basicstyle=\small \ttfamily, % print whole listing small
            keywordstyle=\color{black}\bfseries\underbar,
            ndkeywordstyle=\color{blue}\bfseries,
            identifierstyle=, 					% nothing happens
                commentstyle=\color{cyan} \itshape, % white comments
                stringstyle=\color{red}, 		% typewriter type for strings
                showstringspaces=false 			% no special string spaces
                columns=fullflexible,				%Sans ca, tabsize fonctionne pas
                tabsize=4,									%Nombre d'espace = tab
                %-----Numéros de ligne -----------
                numbers=left,								%Numeros à gauche
                numberstyle=\small \itshape,					%Style des numéros de lignes
                stepnumber=1,								%Fréquence de numérotation
                numbersep=5pt,							%Marge entre les numeros de ligne et le code
                %-----Options diverses-----------
                mathescape=true,						%Possibilite de passer en mode math avec $
                breaklines=true,						%Cassure des lignes trop longues
                %-----Mise en page:frame----------
                frame=shadowbox,						%Frame en relief (ou lines, single)
                %frameround={t}{t}{t}{t},		%BBords arrondis
                rulesepcolor=\color{black}, %Couleur des relief des frame
                framexleftmargin=10mm				%Marge nécessaire pour les numéros de ligne
    }
}

\newcommand{\lstconfigmatlab}
{
    \lstset{language=Matlab}
    \lstset{% general command to set parameter(s)
        %-----Police et coloration syntaxique
            basicstyle=\small \ttfamily, % print whole listing small
            keywordstyle=\color{blue}\bfseries,
            %ndkeywordstyle=\color{blue}\bfseries,
            identifierstyle=, 					% nothing happens
                commentstyle=\color{cyan} \itshape, % white comments
                stringstyle=\color{red}, 		% typewriter type for strings
                showstringspaces=false 			% no special string spaces
                columns=fullflexible,				%Sans ca, tabsize fonctionne pas
                tabsize=4,									%Nombre d'espace = tab
                %-----Numéros de ligne -----------
                numbers=left,								%Numeros à gauche
                numberstyle=\small \itshape,					%Style des numéros de lignes
                stepnumber=1,								%Fréquence de numérotation
                numbersep=5pt,							%Marge entre les numeros de ligne et le code
                %-----Options diverses-----------
                mathescape=true,						%Possibilite de passer en mode math avec $
                breaklines=true,						%Cassure des lignes trop longues
                %-----Mise en page:frame----------
                frame=shadowbox,						%Frame en relief (ou lines, single)
                %frameround={t}{t}{t}{t},		%BBords arrondis
                rulesepcolor=\color{black}, %Couleur des relief des frame
                framexleftmargin=5mm				%Marge nécessaire pour les numéros de ligne
    }
}

\newenvironment{lstpage}
{
    \begin{minipage}[t]{.43\linewidth}
    \begin{lstlisting}[numbers=left]}
{\end{lstlisting}\end{minipage}}


