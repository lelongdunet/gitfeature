%------------ PDF -------------------------------------------------------------
\hypersetup{                                        % Information sur le document
    pdfauthor   = {A Lelong},                     % Auteurs
    pdftitle    = {Git-feature documentation},     % Titre du document
    pdfsubject  = {},      % Sujet
    pdfkeywords = {git, rebase, workflow},                               % Mots-clefs
    pdfcreator  = {Latex},                   % Logiciel qui a crée le document
    pdfproducer = {},                                % Société avec produit le logiciel
    pdfborder         = {0 0 0},                 % Style de bordure : ici, pas de bordure
    plainpages  = true}

%------------ TIKZ --------------------------------------------
%\usetikzlibrary{arrows}%
\usetikzlibrary{arrows,shapes,trees,mindmap,snakes,backgrounds,calc}


\pgfdeclarelayer{background}
\pgfdeclarelayer{foreground}
\pgfsetlayers{background,main,foreground}

%-------------INTERLIGNE-------------------------------------------------
%%\renewcommand{\baselinestretch}{2}
%%Interligne entre deux paragraphes
%\parskip = 5pt
%%Autres package
\usepackage{ulem}
\usepackage{lmodern}           % pour générer de "beaux" fichiers PDF

%-------------ENTTE-ET-PIED-DE-PAGE-------------------------------------------
\pagestyle{fancy}

% Ceci permet d~avoir les noms de chapitre et de section
% en minuscules
%\renewcommand{\chaptermark}[1]{\markboth{#1}{}}%minuscules
\renewcommand{\sectionmark}[1]{\markright{#1}} %minuscules
\renewcommand{\subsectionmark}[1]{} %minuscules

%\lhead[\includegraphics[width=2cm]{imgbin/logo.png}]{\rightmark}
\lhead[gdfgd]{\rightmark}
\rhead[\leftmark]{\includegraphics[width=2cm]{imgbin/logo.png}}


\renewcommand{\headrulewidth}{0.3pt} % filet en haut
\renewcommand{\footrulewidth}{0.3pt} % filet en bas
\addtolength{\headheight}{2pt} % espace pour le filet


%-------------REGLAGE DES MARGES-------------------------------------------------


\usepackage{vmargin}            % redéfinir les marges
% Marge gauche, haute, droite, basse; espace entre la marge et le
% texte à gauche, en haut, à droite, en bas
\setmarginsrb{25mm}{20mm}{20mm}{35mm}{5pt}{8mm}{0pt}{20mm}

