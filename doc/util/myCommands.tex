\usepackage{pgf,pgfcore,tikz}
%\usepackage{tikz-3dplot}
\usepackage{pgfplots}
\usetikzlibrary{arrows,shapes,matrix,snakes,mindmap,backgrounds,plotmarks,calc,automata}%,shadows}
%\usetikzlibrary{shadows}
%%%%%%%%%%%%%%%%%%%%%%%%%%%%%%%%%%%%%%%%%%%%%%%%%%%%%%%%%%%%%%%%%%%%%%%%%%%%%%%%%%%
%%%%%%%%%%%%%%% Commandes pour dessiner des reseaux %%%%%%%%%%%%%%%%%%%%%%%%%%%%%%
%%%%%%%%%%%%%%%%%%%%%%%%%%%%%%%%%%%%%%%%%%%%%%%%%%%%%%%%%%%%%%%%%%%%%%%%%%%%%%%%%%%
\newcommand{\branch}[8]{%
  \pgfmathsetmacro{#3}{cos(#6 r)* #5}
  \pgfmathsetmacro{#4}{sin(#6 r)* #5}
  \draw[thick] (#1,#2) -- (#1 +#3,#2 +#4) node[above,pos=#8] {#7};}

\newcommand{\branchcoil}[8]{%
  \pgfmathsetmacro{#3}{cos(#6 r)* #5}
  \pgfmathsetmacro{#4}{sin(#6 r)* #5}
  \draw[thick,snake=snake,line after snake=0pt, segment aspect=0,segment length=5pt,segment amplitude=1pt] (#1,#2) -- (#1 +#3,#2 +#4) node[above,pos=#8] {#7};}
%%%%%%%%%%%%%%%%%%%%%%%%%%%%%%%%%%%%%%%%%%%%%%%%%%%%%%%%%%%%%%%%%%%%%%%%%%%%%%%%%%%
%%%%%%%%%%%%%%% Commandes pour marquer les longueurs %%%%%%%%%%%%%%%%%%%%%%%%%%%%%%
%%%%%%%%%%%%%%%%%%%%%%%%%%%%%%%%%%%%%%%%%%%%%%%%%%%%%%%%%%%%%%%%%%%%%%%%%%%%%%%%%%%
\newcommand{\drawlen}[7]{%
%
% IN:
%  - x
%  - y
%  - l
%  - topdash
%  - botdash
%  - text
% OUT:
%  - xend
%
  \pgfmathsetmacro{#7}{#1+#3*\coeflen}
%Draw arrow
  \draw[<->,very thick,color=violet!50!green] (#1,#2) -- (#7, #2)
  node [midway,anchor=south,inner sep=2pt, outer sep=1pt]{#6};
\draw[thick,dotted,color=violet!50!green] (#7,#4) -- (#7, #5);}
%%%%%%%%%%%%%%%%%%%%%%%%%%%%%%%%%%%%%%%%%%%%%%%%%%%%%%%%%%%%%%%%%%%%%%%%%%%%%%%%%%%
\newcommand{\drawcriticlen}[7]{%
%
% IN:
%  - x
%  - y
%  - l
%  - topdash
%  - botdash
%  - text
% OUT:
%  - xend
%
  \pgfmathsetmacro{#7}{#1+#3*\coeflen}
%Draw arrow
  \draw[<->,very thick,color=red] (#1,#2) -- (#7, #2)
  node [midway,anchor=south,inner sep=2pt, outer sep=1pt]{#6};
\draw[very thick,dotted,color=red] (#7,#4) -- (#7, #5);}
%%%%%%%%%%%%%%%%%%%%%%%%%%%%%%%%%%%%%%%%%%%%%%%%%%%%%%%%%%%%%%%%%%%%%%%%%%%%%%%%%%%
\newcommand{\drawlostlen}[7]{%
%
% IN:
%  - x
%  - y1
%  - y2
%  - l
%  - topdash
%  - botdash
% OUT:
%  - xend
%
  \pgfmathsetmacro{#7}{#1+#4*\coeflen}
%Draw arrow
  \filldraw[fill=violet!50!green,opacity=.2] (#1,#2) rectangle (#7, #3);
\draw[thick,dotted,color=violet!50!green] (#7,#5) -- (#7, #6);}
%%%%%%%%%%%%%%%%%%%%%%%%%%%%%%%%%%%%%%%%%%%%%%%%%%%%%%%%%%%%%%%%%%%%%%%%%%%%%%%%%%%

%%%%%%%%%%%%%%%%%%%%%%%%%%%%%%%%%%%%%%%%%%%%%%%%%%%%%%%%%%%%%%%%%%%%%%%%%%%%%%%%%%%
%%%%%%%%%%%%%%%%%%%%%%%%%%%%%%%%%%%%%%%%%%%%%%%%%%%%%%%%%%%%%%%%%%%%%%%%%%%%%%%%%%%
\tikzstyle{na} = [baseline=-.5ex]
\tikzstyle{markNode} = [rounded corners, anchor=base]
\newcommand{\tknode}[1]{\tikz[na]\node [coordinate] (#1) {};}
%\pgfplotsset{every axis x label/.append style={xlabel near ticks}}
%\pgfplotsset{every axis y label/.append style={ylabel near ticks}}
\pgfplotsset{xlabel near ticks}
\pgfplotsset{ylabel near ticks}

\newcommand{\inputplot}[1]{\tiny\input{#1}}

%Commandes associées aux notations mathématiques
\newcommand{\tenseur}[1]{\stackrel{\Rightarrow}{#1}}
\newcommand{\M}[1]{\mathbf{#1}}
\newcommand{\V}[1]{\boldsymbol{\mathbf{\underline{#1}}}}
\newcommand{\Ves}[1]{\hat{\V{#1}}}
\renewcommand{\H}{\mathbf{H}}
\newcommand{\T}{\mathbf{T}}
\newcommand{\Prob}{\mathbb{P}}
\newcommand{\E}[1]{\mathbb{E}\left\{{#1}\right\}}
\newcommand{\Ep}[1]{\mathbb{E}\left({#1}\right)}
\newcommand{\C}{\mathscr{C}}
\newcommand{\diag}{\mbox{\textbf{diag}}}
\newcommand{\floor}[1]{\left\lfloor#1\right\rfloor}
\newcommand{\modulo}[2]{\overline{#1}^{#2}}
\newcommand{\SE}[1]{\tilde{#1}}
\newcommand{\Dist}{\mathbf{\mathcal{D}}}
\newcommand{\Regr}{(-)}
\newcommand{\Progr}{(+)}
\newcommand{\thclean}{\epsilon_{CLEAN}}
\newcommand{\resp}{\textit{resp. }}
\newcommand{\one}{\mathds{1}}
\newcommand{\R}{\mathbb{R}}
\newcommand{\Rp}{\mathbb{R}^+}
\newcommand{\erf}{\mbox{erf}}
\newcommand{\Ref}{\mbox{\tiny ref}}
\newcommand{\Cur}{\mbox{\tiny cur}}
\newcommand{\mx}[1]{\mathbf{\bm{#1}}} % Matrix command
\newcommand{\vc}[1]{\mathbf{\bm{#1}}} % Vector command
\def\eqdef {\buildrel \rm \vartriangle \over =}    % Egal par definition.
\def\eqas  {\buildrel \rm a.s. \over =}   % Egal a.s.
\def\toas  {\buildrel \rm a.s. \over \to} % ---> a.s.

\newcommand{\HalfLine}{\rule[1ex]{0.5\textwidth}{0.1mm}}

\newcommand{\TriRelief}{\ding{226}}% : Fleche en relief


\newcommand{\revAdd}[1]{#1}        %Disable
%\newcommand{\revAdd}[1]{\emph{#1}}%Enable
\newcommand{\IDX}[1]{\textit{#1}}
\newcommand{\refEq}[1]{(\ref{#1})}

\definecolor{lightgreen}{rgb}{0.8,1.0,0.8}
\definecolor{lightred}{rgb}{1.0,0.8,0.8}

\newcommand*\Up{\textcolor{green}{%
  \ensuremath{\blacktriangle}}}
\newcommand*\Down{\textcolor{red}{%
  \ensuremath{\blacktriangledown}}}
\newcommand*\Const{\textcolor{darkgray}%
  {\textbf{--}}}

%% Hyphnenation
%\hyphenation{échantillon-nage}



\newenvironment{itemizeSpec}{%
\begin{itemize}%[topsep=5pt,partopsep=5pt]
\setlength{\itemsep}{2pt}
}
{%
\end{itemize}
\vspace{5pt}
}
% \begin{itemize}
%   \setlength{\itemsep}{5pt}
%   \setlength{\parskip}{10pt}
%   \setlength{\parsep}{10pt}
% }
% {%
% \end{itemize}}

% GanttHeader setups some parameters for the rest of the diagram
% #1 Width of the diagram
% #2 Width of the space reserved for task numbers
% #3 Width of the space reserved for task names
% #4 Number of months in the diagram
% In addition to these parameters, the layout of the diagram is influenced
% by keys defined below, such as y, which changes the vertical scale
\def\GanttHeader#1#2#3#4{%
\shorthandoff{:;!?}
 \pgfmathparse{(#1-#2-#3)/#4}
 \tikzset{y=7mm, task number/.style={left, font=\bfseries},
     task description/.style={text width=#3,  right, draw=none,
           font=\sffamily, xshift=#2,
           minimum height=2em},
     gantt bar/.style={draw=black, rounded corners, fill=blue!30},
     help lines/.style={draw=black!30, dashed},
     x=\pgfmathresult pt
     }
  \def\totalmonths{#4}
  \def\yscale{0.82}
  \node (Header) [task description] at (0,0) {\textbf{\large Tâches}};
  \begin{scope}[shift=($(Header.south east)$)]
    \fill[color=red!10, rounded corners] (0,0) rectangle +(\totalmonths+0.2, 0.6);
    \foreach \x in {1,...,#4}
      \node[above] at (\x,0) {\footnotesize\x};
 \end{scope}
}

% This macro adds a task to the diagram
% #1 Number of the task
% #2 Task's name
% #3 Starting date of the task (month's number, can be non-integer)
% #4 Task's duration in months (can be non-integer)
\def\Task#1#2#3#4{%
\node[task number] at ($(Header.west) + (0, -#1*\yscale)$) {#1};
\node[task description] at (0,-#1*\yscale) {#2};
\begin{scope}[shift=($(Header.south east)$)]
  %\draw (0,-#1*\yscale) rectangle +(\totalmonths, 1);
  \fill[color=red!10, rounded corners] (0,-#1*\yscale) rectangle +(\totalmonths+0.2, 1);
  \foreach \x in {1,...,\totalmonths}
    \draw[help lines] (\x,-#1*\yscale) -- +(0,1);
  %\filldraw[gantt bar] ($(#3, -#1+0.2)$) rectangle +(#4,0.6);
  \shadedraw[gantt bar] ($(#3, -#1*\yscale+0.2)$) rectangle +(#4,0.6);
\end{scope}
}

% This macro adds a task to the diagram
% #1 Number of the task
% #2 Task's name
% #3 Starting date of the task first part
% #4 Task's duration in months first part
% #5 Starting date of the task second part
% #6 Task's duration in months second part
\def\DualTask#1#2#3#4#5#6{%
\node[task number] at ($(Header.west) + (0, -#1*\yscale)$) {#1};
\node[task description] at (0,-#1*\yscale) {#2};
\begin{scope}[shift=($(Header.south east)$)]
  %\draw (0,-#1*\yscale) rectangle +(\totalmonths, 1);
  \fill[color=red!10, rounded corners] (0,-#1*\yscale) rectangle +(\totalmonths+0.2, 1);
  \foreach \x in {1,...,\totalmonths}
    \draw[help lines] (\x,-#1*\yscale) -- +(0,1);
  %\filldraw[gantt bar] ($(#3, -#1+0.2)$) rectangle +(#4,0.6);
  \shadedraw[gantt bar] ($(#3, -#1*\yscale+0.2)$) rectangle +(#4,0.6);
  \shadedraw[gantt bar] ($(#5, -#1*\yscale+0.2)$) rectangle +(#6,0.6);
\end{scope}
}


